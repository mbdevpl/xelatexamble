
% Lorem ipsum dolor sit amet, consectetuer adipiscing elit...
\usepackage{lipsum}
\setlipsumdefault{1}

% supress 'xparse/redefine-command' warnings
\usepackage[log-declarations=false]{xparse}

% for \iflinux and \ifwindows checks
\usepackage{ifplatform}

% support for chinese, japanese, korean
\usepackage{xeCJK}
% examples: Meiryo, Meiryo UI, MS Mincho, MS PMincho, MS Gothic, MS PGothic, MS UI Gothic
% incomplete examples: SimSun, SimHei, FangSong, KaiTi
\iflinux
% set default font for xeCJK
\setCJKmainfont{IPAPMincho}
% set sans font for xeCJK
\setCJKsansfont{IPAMincho}
% set monospace font for xeCJK
\setCJKmonofont{IPAPMincho}
\else\ifwindows
\setCJKmainfont{MS PMincho}
\setCJKsansfont{Meiryo UI}
\setCJKmonofont{Meiryo}
\else
\setCJKmainfont{IPAPMincho}
\setCJKsansfont{IPAMincho}
\setCJKmonofont{IPAPMincho}
\fi
\fi

% recommended when using English hyphenation with babel
\usepackage{csquotes}

% hyphenation
\usepackage[english]{babel}

% for proper placement of floating figures
\usepackage{float}

%use \begin{wrapfigure}[lineheight]{position}{width} for wrapping text around figures
\usepackage{wrapfig}

% captions for figures and tables
\usepackage{caption}

% use \begin{mdframed} ... \end{mdframed} to add frames around things
\usepackage{mdframed}

% colors
\usepackage{color}
% more colors, new colors with \definecolor{name}{model}{color-spec}
% e.g. \definecolor{light-gray}{gray}{0.95} or \definecolor{orange}{RGB}{255,127,0}
% and for colored text use \textcolor{declared-color}{text}
\usepackage[usenames,dvipsnames,svgnames,table,x11names]{xcolor}

%big braces and other useful symbols
\usepackage{amssymb}

% some mathematical symbols
% conflicts with package program
\usepackage{amsmath}

% fancy matrix environments
\usepackage{mathtools}

% this needs to be loaded before unicode-math package
\usepackage{currfile}

% needed for fancy math fonts
\usepackage[warnings-off={mathtools-colon,mathtools-overbracket}]{unicode-math}

% change math mode font to a better one, needs package asana-math
\setmathfont[version=Asana]{Asana Math}
%\setmathfont[version=Cambria]{Cambria Math}
%\setmathfont[version=Xits]{XITS Math}
\setmathfont[version=LatinModern]{Latin Modern Math} % package: lm-math
\mathversion{Asana}
%\mathversion{Cambria}
%\mathversion{xits}

% use \begin{comment} ... \end{comment} for multiline comments
\usepackage{comment}

% use \begin{multicols}{#} for # columns
\usepackage{multicol}

% tables with flexible column widths using 'X'
\usepackage{tabularx}

% use \includegraphics[scale=1.00]{file.jpg} for images
% more parameters: width=\textwidth,height=0.8\textheight,keepaspectratio
\usepackage{graphicx}

% drawing versatile vector graphics with \begin{tikzpicture} ... \end{tikzpicture}
\usepackage{tikz}
\usetikzlibrary{positioning}
\usetikzlibrary{arrows,automata}
\usetikzlibrary{shapes}

% charts and plots in any form
\usepackage{pgfplots}
\pgfplotsset{compat=1.9}

% to get plot data from CSV
\usepackage{csvsimple}

% inline file contents using \begin{filecontents*}{benchmarkb.csv} ... \end{filecontents*}
\usepackage{filecontents}

%use \begin{minted}[mathescape,linenos,numbersep=5pt,gobble=0,framesep=2mm]{c++}
\usepackage{minted}
\iflinux
% run 'pythontex document.tex' after first build
\usepackage{pythontex}
\setpythontexlistingenv{pythontexlisting}
\fi

% page x of y with: \cfoot{\thepage{} of \pageref{LastPage}}
\usepackage{lastpage}

% to handle multiple footnotes in title and authors area
\usepackage[multiple]{footmisc}

% also needed for footers
\usepackage{fancyhdr}

% insert web addresses using \url{http://domain.com/}
\usepackage{url}

% use \addbibresource[datatype=bibtex]{file.bib} to add bibliography
% some more options: ,bibstyle=authortitle,url=false,isbn=false,autocite=square,autocite=superscript,
\usepackage[backend=biber,hyperref=true,backref=true,style=numeric,sorting=none]{biblatex}
